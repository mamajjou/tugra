\documentclass[a4paper,12pt]{article}
\usepackage{amsmath}
\usepackage{xcolor}
\usepackage{geometry}
\geometry{margin=2.2cm}

\begin{document}

\begin{center}
    {\LARGE\bfseries Von der Normalform zur Scheitelpunktsform}
\end{center}

\vspace{1em}

Wir beginnen mit der Parabel in Normalform:
\[
f(x) = \textcolor{blue}{a}x^2 + \textcolor{red}{b}x + \textcolor{green}{c}.
\]

\section*{1. \textcolor{blue}{a} ausklammern}
\[
f(x) = \textcolor{blue}{a}\Bigl(x^2 + \frac{\textcolor{red}{b}}{\textcolor{blue}{a}} x \Bigr)
+ \textcolor{green}{c}.
\]

\section*{2. Quadratische Ergänzung}
Berechne:
\[
\left(\frac{\textcolor{red}{b}}{2\textcolor{blue}{a}}\right)^2.
\]

Führe sie im Inneren ein und ziehe sie wieder ab:
\[
f(x)
= \textcolor{blue}{a}\Bigl(
x^2 + \frac{\textcolor{red}{b}}{\textcolor{blue}{a}}x
+ \left(\frac{\textcolor{red}{b}}{2\textcolor{blue}{a}}\right)^2
- \left(\frac{\textcolor{red}{b}}{2\textcolor{blue}{a}}\right)^2
\Bigr)
+ \textcolor{green}{c}.
\]

\section*{3. Zum Quadrat zusammenfassen}
\[
x^2 + \frac{\textcolor{red}{b}}{\textcolor{blue}{a}}x
+ \left(\frac{\textcolor{red}{b}}{2\textcolor{blue}{a}}\right)^2
=
\left(x + \frac{\textcolor{red}{b}}{2\textcolor{blue}{a}}\right)^2.
\]

Damit:
\[
f(x) =
\textcolor{blue}{a}\left(x + \frac{\textcolor{red}{b}}{2\textcolor{blue}{a}}\right)^2
- \textcolor{blue}{a}\left(\frac{\textcolor{red}{b}}{2\textcolor{blue}{a}}\right)^2
+ \textcolor{green}{c}.
\]

\section*{4. Vereinfachen}
\[
- \textcolor{blue}{a}\left(\frac{\textcolor{red}{b}}{2\textcolor{blue}{a}}\right)^2
= -\frac{\textcolor{red}{b}^2}{4\textcolor{blue}{a}}.
\]

\section*{5. Scheitelpunktsform}
\[
\boxed{
f(x)
= \textcolor{blue}{a}
\left(x + \frac{\textcolor{red}{b}}{2\textcolor{blue}{a}}\right)^2
+ \left(
\textcolor{green}{c} -
\frac{\textcolor{red}{b}^2}{4\textcolor{blue}{a}}
\right)}
\]

\section*{Scheitelpunkt}
\[
S\left(
-\frac{\textcolor{red}{b}}{2\textcolor{blue}{a}},\;
\textcolor{green}{c} - \frac{\textcolor{red}{b}^2}{4\textcolor{blue}{a}}
\right).
\]

\end{document}