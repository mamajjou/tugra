\documentclass[12pt]{article}
\usepackage{amsmath, amssymb}
\usepackage[margin=1.7cm]{geometry}
\usepackage{hyperref}
\usepackage{graphicx}
\usepackage{tikz}
\usepackage{enumitem}
\setlength{\parindent}{0pt}
\setlist{itemsep=0.35em, topsep=0.35em}

\title{PQ-Formel Übungen}
\author{}
\date{05.12.2025}

\begin{document}
\maketitle

\section*{PQ-Formel}
Für jede Gleichung der Form \(x^{2} + px + q = 0\) gilt:
\[
  x_{1,2} = -\frac{p}{2} \pm \sqrt{\left(\frac{p}{2}\right)^{2} - q}.
\]
Notiere immer zuerst \(p\) und \(q\), setze sie in die Formel ein und rechne den Ausdruck unter der Wurzel (Diskriminante) aus.

\textbf{Hinweis.} Begründe alle Schritte der Rechnung. Exakte Werte (ggf. mit Wurzel) reichen; nicht runden. Schreibe zu jeder Aufgabe kurz auf, welches \(p\) und welches \(q\) du einsetzt.

\section*{Aufgabe 1: Direkt einsetzen (schöne Zahlen)}
Berechne die Nullstellen. Setze \(p\) und \(q\) sauber ein und gib beide Lösungen an.
\begin{enumerate}
  \item \(y = x^{2} - 8x + 12\)
  \item \(y = x^{2} + 5x + 4\)
  \item \(y = x^{2} - 3x - 10\)
  \item \(y = x^{2} - 7x + 10\)
\end{enumerate}

\section*{Aufgabe 2: Erst normieren, dann Nullstellen}
Bringe jede Gleichung zuerst in die Form \(x^{2} + px + q = 0\) (durch Teilen oder Ausklammern), lies dann \(p\) und \(q\) ab und berechne die Nullstellen.
\begin{enumerate}
  \item \(2x^{2} + 6x - 8 = 0\)
  \item \(-3x^{2} + 12x - 12 = 0\)
  \item \(5x^{2} - 5x - 10 = 0\)
\end{enumerate}
\newpage
\section*{Aufgabe 3: Diskriminante deuten}
Berechne die Nullstellen mit der PQ-Formel und formuliere dazu, ob die Parabel die \(x\)-Achse schneidet, berührt oder nicht schneidet.
\begin{enumerate}
  \item \(y = x^{2} - 10x + 25\)
  \item \(y = x^{2} + 2x + 5\)
  \item \(y = x^{2} - 4x - 45\)
\end{enumerate}

\section*{Aufgabe 4: Wurzelwerte stehen lassen}
Rechne die Nullstellen aus und lasse Wurzelterme stehen (nicht als Dezimalzahl runden).
\begin{enumerate}
  \item \(y = x^{2} - 6x + 2\)
  \item \(y = x^{2} + 4x - 7\)
  \item \(y = 3x^{2} - 3x - 2\) \quad (Tipp: zuerst durch \(3\) teilen)
\end{enumerate}

\section*{Aufgabe 5: Sortiere nach Lösungstyp}
Ordne den folgenden Parabeln zu, ob sie zwei verschiedene Nullstellen, eine doppelte Nullstelle oder keine reellen Nullstellen besitzen. Begründe kurz mit dem Wert unter der Wurzel.
\begin{enumerate}
  \item \(y = x^{2} + 4x + 4\)
  \item \(y = x^{2} + 7x + 6\)
  \item \(y = x^{2} + x + 3\)
  \item \(y = x^{2} - 11x + 30\)
\end{enumerate}

\end{document}
