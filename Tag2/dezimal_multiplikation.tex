\documentclass[12pt]{article}
\usepackage{amsmath, amssymb}
\usepackage{hyperref}
\usepackage{graphicx}
\setlength{\parindent}{0pt}

\title{Übung: Zehntel mit Kopfrechnung und Komma-Setzung}
\author{}
\date{04.12.2025}

\begin{document}
\maketitle

\textbf{Hinweis.} Schreibe jeden Schritt auf: zuerst das Produkt der ganzen Zahlen, dann die Anzahl der Nachkommastellen. Kontrolliere mit Überschlagsrechnung, ob die Größenordnung passt.

\section*{Aufgabe 1: Quadrate von Zehnteln}
Berechne und notiere die Zwischenschritte (z.\,B.\ \(0{,}6^{2} = 6^{2} / 10^{2}\)).
\[
0{,}2^{2},\quad 0{,}3^{2},\quad 0{,}4^{2},\quad 0{,}5^{2},\quad 0{,}6^{2},\quad 0{,}7^{2},\quad 0{,}8^{2},\quad 0{,}9^{2}
\]

\section*{Aufgabe 2: Produkt aus zwei Zehnteln}
Rechne schriftlich oder mit klaren Zwischenschritten:
\[
0{,}4 \cdot 0{,}7,\qquad 0{,}3 \cdot 0{,}9,\qquad 0{,}5 \cdot 0{,}8,\qquad 0{,}6 \cdot 0{,}6.
\]
Notiere jeweils:
\begin{itemize}
  \item das Produkt der Zähler (ohne Komma),
  \item die Gesamtzahl der Nachkommastellen,
  \item das Endergebnis.
\end{itemize}

\section*{Aufgabe 3: Gemischte Größenordnungen}
Rechne, achte auf das Komma:
\[
1{,}2 \cdot 0{,}5,\qquad 2{,}5 \cdot 0{,}4,\qquad 3{,}6 \cdot 0{,}3,\qquad 4{,}8 \cdot 0{,}2.
\]
Mache einen Überschlag (z.\,B.\ \(1{,}2 \cdot 0{,}5 \approx 1{,}2 / 2 \approx 0{,}6\)) und vergleiche mit deinem Ergebnis.

\section*{Aufgabe 4: Lücken füllen}
Trage die Ergebnisse ein. Jede Aufgabe hat zwei Nachkommastellen im Ergebnis.
\[
\begin{array}{c|c}
0{,}2 \cdot 0{,}2 & \_\_\,\_\_ \\ \hline
0{,}3 \cdot 0{,}5 & \_\_\,\_\_ \\ \hline
0{,}7 \cdot 0{,}9 & \_\_\,\_\_ \\ \hline
0{,}8 \cdot 0{,}6 & \_\_\,\_\_ \\ \hline
0{,}9 \cdot 0{,}4 & \_\_\,\_\_
\end{array}
\]

\section*{Aufgabe 5: Eigene Kontrolle}
Erfinde zwei weitere Aufgaben mit Zehnteln (z.\,B.\ \(0{,}a \cdot 0{,}b\)). Rechne sie aus und schreibe dazu, wie viele Nachkommastellen du erwartest und warum.

\end{document}
