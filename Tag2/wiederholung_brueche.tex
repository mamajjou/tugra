\documentclass[12pt]{article}
\usepackage{amsmath, amssymb}
\usepackage{hyperref}
\usepackage{graphicx}
\usepackage{tikz}
\setlength{\parindent}{0pt}

\title{Wiederholung: Bruchrechnen und Parabeln}
\author{}
\date{04.12.2025}

\begin{document}
\maketitle

\textbf{Hinweis.} Zeige alle Zwischenschritte. Achte auf Vorzeichen, besonders bei Doppelminus und beim Kürzen von Brüchen.

\section*{1. Brüche mit Vorzeichen}
Vereinfache (gemeinsamer Nenner, dann kürzen):
\[
  -\frac{3}{4} + \frac{5}{6}
  \qquad\text{und}\qquad
  -\frac{2}{3} -\!\left(-\frac{1}{2}\right).
\]

\section*{2. Funktionswerte mit Brüchen}
Für \(f(x) = \tfrac{1}{2}x^{2} - \tfrac{3}{2}x + 1\) berechne \(f(x)\) und vereinfache:
\[
  x = 0,\quad x = 1,\quad x = 3.
\]

\section*{3. Scheitelpunktform mit Brüchen}
Bringe \(y = \tfrac{1}{2}x^{2} + x - \tfrac{3}{2}\) durch quadratische Ergänzung in Scheitelpunktform. Bestimme Scheitelpunkt und Symmetrieachse.
\newpage
\section*{4. PQ-Formel mit Brüchen}
Die PQ-Formel für \(x^{2} + px + q = 0\) lautet
\[
  x_{1,2} = -\frac{p}{2} \pm \sqrt{\left(\frac{p}{2}\right)^{2} - q}.
\]
Wende sie auf \(x^{2} - \tfrac{5}{2}x + 1 = 0\) an. Gib die Nullstellen exakt an und prüfe das Vorzeichen unter der Wurzel.

\section*{5. Analyse einer Parabel mit Brüchen}
Gegeben \(g(x) = -\tfrac{2}{3}x^{2} + \tfrac{4}{3}x - \tfrac{1}{3}\).
\begin{enumerate}
  \item Bestimme Scheitelpunkt und Symmetrieachse.
  \item Berechne den \(y\)-Achsenabschnitt.
  \item Entscheide mit der Diskriminante, ob \(x\)-Achsenabschnitte existieren (rechne exakt mit Brüchen).
\end{enumerate}

\end{document}
